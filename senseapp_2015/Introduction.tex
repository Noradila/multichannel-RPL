\section{Introduction}
\label{sec:introduction}

%- what problem are we solving?
Low power radios such as IEEE 802.15.4 are a relatively short range transmission standard radio technology in the 2.4Ghz band.  The standard allows transmission to occur on several different channels within this band.  Unfortunately, the
channels used by this technology often suffer interference, for example, from WiFi and Bluetooth.  Wireless Sensor Networks (WSN) need to be
able to operate reliably in the presence of such interference.  The IETF standard RPL is a routing protocol for WSN based on IEEE 802.15.4 that allows nodes to self-organise a communicating network of neighbouring nodes.  In
its most usual mode of operation RPL operates with a layer two which uses only a single channel.  In this paper we develop a cross-layer multi-channel
protocol which allows a centralised intelligence to determine which channels each node should listen on and which ensures that their neighbours send on the correct channel.  The protocol also introduces a probing phase that checks whether assigned channels are free of interference.  This protocol is tested using a two-hop colouring protocol that ensures nodes located within two hops of each other in the network are listening on different channels (that should reduce interference between physically proximate nodes trying to communicate on the same channel).  

%Unfortunately, the frequency band is shared with WiFi and Bluetooth which causes problem for WSN that require minimal packet loss, interference and delay. Problems arise as all 16 channels overlap with channels used by WiFi. However, the overlapping problem can be solved if WiFi only uses the European non-overlapping channel set which are channels 1, 7 and 13, leaving channels 15, 16, 21 and 22 to be used by 802.15.4 \cite{ieee_2012}. Another solution is to use multichannel.

%why multichannel?
%- multichannel enables communication between more than one node to occur simultaneously and in a collision free manner
Multichannel communication in wireless networks can alleviate the effects of interference which, as a result, can improve the network efficiency and stability, link reliability and minimise latency. It also enables communication between physically proximate nodes to occur simultaneously without the risk of collision. However, not all channels are free from interference, thus, the need to hop to another channel when the quality of the channel deteriorates. Two commonly used types of channel hopping \cite{watteyne} are blind channel hopping and whitelisting. In blind channel hopping, nodes choose from all available channels. Whitelisting on the other hand, filters out those channels that may have bad interference properties. Many studies make use of channel whitelisting such as \cite{watteyne} which claimed that channels 11, 15, 25 and 26 are free from Wi-Fi, \cite{wu}. Chrysso \cite{chrysso} uses channel 11, 14, 20, 22 and 26, and MiCMAC \cite{micmac} uses channel 15, 20, 25 and 26. It is clear that it 
is impossible to find a single channel guaranteed free from interference and there is no consensus on the best channel to use. Our work takes into account all available channels to utilise the spectrum and checks the condition of the channels before hopping to avoid those channels with interference.

%-what has already happened and what are the problems with that?


%This might be due to the complexity of the solutions to be implemented. [RGC -- I don't think you should speculate like this]

%-mention there is no separate control channel (nodes don't need to keep on switching to get the control message - DIO, DAO etc message)
%- channel reuse?
%- collision free communication?
%- deafness problem?
%- when node sleep/awake, it will be on the listening channel. if it needs to send, it will switch to the transmitting channel and back to listening channel after sending. No need for channel negotiation - only do when get message from LPBR telling it to switch

Several previous studies have developed a multichannel MAC layer but, despite the potential benefits none are yet widely implemented.  The usual focus
is on MAC layers that operate in an autonomous fashion.  This paper focuses instead on a cross-layer multi-channel model where a centralised controller can make and communicate decisions about channels and this decision is implemented by the MAC layer which also provides feedback when a channel is subject to interference using a probing phase. The nodes are given different channels by a centralised contoller behind the Low Power Border Router (LPBR).  This means that the mechanism for assigning nodes to channels can be aware of the entire topology and can use more advanced algorithms to choose which channels are assigned to which nodes.  Nodes are given a listening channel and their neighbours must send to them on
that channel.  In other words, a node listens on a single channel but sends on many channels.
The control messages are sent to the nodes on their usual listening channel as a unicast which eliminates the need for a separate control channel.
The changes of channel occur after the RPL topology set up phase.  We show that this allows the network to avoid channels with interference and we
demonstrate in simulation that this greatly reduces the effects of interference.  While our protocol has an overhead in terms of packets sent and in terms of set up time, the number of packets sent is still low overall and during the multi-channel set up phase the system is still capable of
sending traffic, thus the network is still fully functional after RPL set up but during our protocol's channel changes.
%In Contiki, a fast turnaround time is supported where the channel switching delay of 128$\mu$s is negligible.  [RGC note, I don't know what this means]

The rest of the paper is organised as follows: Section \ref{sec:introduction} presents related work to multichannel protocols. Section III describes the key idea of our proposed protocol and the high-level design and the implementation of the protocol in Contiki. We describe and evaluate the experimental results in Section IV. Finally, we conclude in Section V.

%\input{ewsn_adila_2014.bbl}
