\section{Introduction}

%- what problem are we solving?
Low power radios such as IEEE 802.15.4 is a relatively short range transmission standard radio technology in the 2.4Ghz band. Unfortunately, the frequency band is shared with WiFi and Bluetooth which cause problem for WSN that require minimal packet loss, interference and delay. The problem arise as all 16 channels overlap with channels used by WiFi. However, the overlapping problem can be solved if WiFi only uses the European non-overlapping channel set which are channels 1, 7 and 13, leaving channels 15, 16, 21 and 22 to be used by 802.15.4 \cite{ieee_2012}. Another solution is to use multichannel.

%why multichannel?
%- multichannel enables communication between more than one node to occur simultaneously and in a collision free manner
Multichannel communication in wireless networks can alleviate the effects of interference which as a result, improve the network efficiency and stability, link reliability and minimise latency. It also enables communication between nodes to occur simultaneously without the risk of collision. However, not all channels are free from interference, thus, the need to hop to another channel when the quality of the channel deteriorates. There are two types of channel hopping \cite{watteyne}, blind channel hopping and whitelisting. In blind channel hopping, the node will hops over all available channel. Whitelisting on the other hand, filters out the worst channel. Many studies make use of channel whitelisting such as \cite{watteyne} claimed that channel 11, 15, 25 and 26 are free from Wi-Fi, \cite{wu} channel 11, 19 and 25, Chrysso \cite{chrysso} uses channel 11, 14, 20, 22 and 26, and MiCMAC \cite{micmac} uses channel 15, 20, 25 and 26. From these studies, it can be concluded that it is impossible to determine a single channel that is not affected by interference. Our proposed work takes into account all available channels to utilise the spectrum and checks the condition of the channels before hopping as interference varies over time.

%-what has already happened and what are the problems with that?

There are many studies that were done in multichannel where most of them concentrated on using multichannel MAC layer. Despite there are many multichannel MAC layer that are available in the literature, multichannel is still not widely implemented even though it has many potential benefits for wireless networks. This might be due to the complexity of the solutions to be implemented. 

%-mention there is no separate control channel (nodes don't need to keep on switching to get the control message - DIO, DAO etc message)
%- channel reuse?
%- collision free communication?
%- deafness problem?
%- when node sleep/awake, it will be on the listening channel. if it needs to send, it will switch to the transmitting channel and back to listening channel after sending. No need for channel negotiation - only do when get message from LPBR telling it to switch

In this paper, we propose a multichannel RPL variant. As most multichannel is implemented in MAC layer, our work concentrates on the application layer. The nodes are given different channels by the Low Power and lossy network Border Router (LPBR) after the topology is formed to avoid collision in a single channel. LPBR has the knowledge of the whole topology which enables it to assign channel to the nodes. As a result, synchronisation is not require. The nodes communicate on the transmission channel and are always listening on their listening channel. The control messages are sent to the nodes on their listening channel as unicast which eliminate the need for a separate control channel. In Contiki, a fast turnaround time is supported where the channel switching delay of 128$\mu$s is negligible.

The rest of the paper is organised as follows: Section II presents related work to multichannel protocols. Section III describes the key idea of our proposed protocol and the high-level design and the implementation of the protocol in Contiki. We describe and evaluate the experimental results in Section IV. Finally, we conclude in Section V.

%\input{ewsn_adila_2014.bbl}