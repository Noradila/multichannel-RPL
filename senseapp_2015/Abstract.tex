This paper proposes a new multi-channel tree building protocol for ad-hoc sensor networks. %Low power radios such as IEEE 802.15.4 are a relatively short range transmission standard radio technology in the 2.4Ghz band. Unfortunately, the frequency band is shared with WiFi and Bluetooth which cause problem for Wireless Sensor Networks that require minimal packet loss, interference and delay. 
Our protocol alleviates the effect of interference which results in improved network efficiency and stability, link reliability and minimised latency. 
        Our proposal takes into account all available channels to utilise the spectrum. It checks the condition of all the channels before deciding on a channel to switch into. The successful transmission rate of the channels are stored externally from the sensors which can be accessed when require. This information is used to limit the channels to be considered when channel switching is invoked. The channel that is selected is checked for any changes in its condition that might had taken place after it was checked previously before committing to the channel. The results and decisions are informed to the other nodes to update their neighbour table. We use two-hop colouring protocol to avoid collision. 
	By basing our protocol in routing protocol for low power and lossy networks (RPL), packets can be sent to the destination the same way as a single channel RPL but with less loss. 
	%RPL is a gradient based routing protocol forming any-to-any routing for low power IPv6 networks rooted at a single destination called the Low Power and lossy network Border Router (LPBR) with no cycles. The topology is formed by choosing the minimum rank which is the distance from the node to the root based on the minimum expected transmission count (ETX) metric. 
	All nodes are battery operated except for the LPBR. This enables a centralised channel switching processes at the LPBR. The channel switching processes take place after the topology is formed to further improve the transmission rate on the best paths.
	We implement and evaluate our solution using the Contiki framework. Our experimental results demonstrate an increased resilience to interference, significant higher throughput making better use of the total available spectrum and link stability. 
