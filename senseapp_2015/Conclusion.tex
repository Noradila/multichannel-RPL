\section{Conclusion}
\label{sec:conclusion}
%//conclusion and future work

%//"Work is continuing to develop this protocol.  The next stages are 1) to improve the interference model in testing to a multiple channel interference model; 2) To move the implemenation to real hardware and 3) To allow continual updates on packet loss for each node so that channels can be changed dynamically when interference occurs."

We presented Multichannel RPL, a centralised cross layer channel change protocol as an extension to the existing RPL with ContikiMAC. Our protocol mitigates the effect of interference by avoiding the affected channel through probing when deciding a new channel. The results from the simulation show that our protocol avoided the interfered channel to maintain a high throughput over time. 

We are continuing with the work to further develop this protocol. The next stages that we plan to pursue is to improve the interference model that we used in testing to cover multiple channels interference. We plan to replicate the interference model to closely represent the real world where interference happen at many channels. We also plan to test our implementation on real hardware and to allow continual updates on packet loss for each node so that channels can be changed dynamically when interference occurs.

% The nature of our protocol that decides/run the decision making to the application layer gives us more opportunity to add the decision complexity - intelligence? and do real time channel checking. Our protocol maintains high reliability during heavily interfered periods where  ContikiMAC showed a low throughput of //////result?.

