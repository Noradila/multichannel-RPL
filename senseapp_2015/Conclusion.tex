\section{Conclusion}
\label{sec:conclusion}
%//conclusion and future work

%//"Work is continuing to develop this protocol.  The next stages are 1) to improve the interference model in testing to a multiple channel interference model; 2) To move the implemenation to real hardware and 3) To allow continual updates on packet loss for each node so that channels can be changed dynamically when interference occurs."

We presented MCRP, a centralised cross-layer protocol that mitigates the effect of interference by avoiding affected channels and allows better spectrum usage by trying to move nearby nodes to listen on different channel. The interference avoidance is through probing when moving to a new channel. The results from the simulation showed that our protocol avoids channels with interference and hence avoids packet loss.  We also showed that this system works best when trying to avoid extreme interference

Future work is ongoing to develop the protocol. The next stages that we plan to pursue is to improve the interference model that we used in testing to cover multiple interference channels replicating the real world environment. The protocol will be tested against competing multi-channel protocols such as MiCMAC. We also plan to test our implementation on real hardware.  Finally we will allow nodes to update the LPBR on packet loss experienced in order that changes to interference patterns in the network can be reacted to.

% The nature of our protocol that decides/run the decision making to the application layer gives us more opportunity to add the decision complexity - intelligence? and do real time channel checking. Our protocol maintains high reliability during heavily interfered periods where  ContikiMAC showed a low throughput of //////result?.

