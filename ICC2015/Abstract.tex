This paper proposes a new centralised multi-channel tree building protocol for ad-hoc sensor networks. The protocol alleviates the effect of interference which results in improved network efficiency and stability, link reliability and minimised latency. 
        The proposal takes into account all available channels to utilise the spectrum and aims to use the spectrum efficiently by transmitting on several channels. The protocol detects which channels suffer interference and changes away from those channels. The algorithm for channel selection is a two-hop colouring protocol that reduces the chances of nearby nodes to transmit on the same channel. 
        All nodes are battery operated except for the low power border router (LPBR). This enables a centralised channel switching process at the LPBR. The protocol is inspired by the routing protocol for low power and lossy networks (RPL). In its initial phase, the protocol uses RPL's standard topology formation to create an initial working topology and then seeks to improve this topology by switching channels.
        The implementation and evaluation of the protocol is performed using the Contiki framework.
The experimental results demonstrate an increased resilience to interference and significant higher throughput making better use of the total available spectrum and link stability.