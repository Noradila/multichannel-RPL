This paper proposes a new centralised multi-channel tree building protocol for ad-hoc sensor networks. Our protocol alleviates the effect of interference which results in improved network efficiency and stability, link reliability and minimised latency. 
	Our proposal takes into account all available channels to utilise the spectrum and aims to use the spectrum efficiently by transmitting on several channels. The protocol detects which channels suffer interference and changes away from those channels. We use two-hop colouring protocol to avoid collision. 
In order to maintain a link when a node changes its channel, all neighbours must be informed of the change.
	All nodes are battery operated except for the low power border router (LPBR). This enables a centralised channel switching process at the LPBR. Our protocol is inspired by the routing protocol for low power and lossy networks (RPL). In its initial phase, the protocol uses RPL's standard topology formation to create an initial working topology and then seeks to improve this topology by switching channels.
	We implement and evaluate our solution using the Contiki framework. Our experimental results demonstrate an increased resilience to interference, and significant higher throughput making better use of the total available spectrum and link stability. 



%This paper proposes a new centralised multi-channel tree building protocol for ad-hoc sensor networks. Our protocol alleviates the effect of interference which results in improved network efficiency and stability, link reliability and minimised latency. 

%Our proposal takes into account all available channels to utilise the spectrum. Our protocol aims to use the spectrum efficiently by transmitting on several channels. The protocol detects which channels suffer interference and changes away from those channels.

%%%
%It checks the condition of all the channels before deciding on a channel to switch into. The successful transmission rate of the channels are stored externally from the sensors which can be accessed when require. 
%This information is used to limit the channels to be considered when channel switching is invoked. 
%The channel that is selected is checked for any changes in its condition that might had taken place after it was checked previously before committing to the channel. 
%The results and decisions are informed to the other nodes to update their neighbour table. 
%%%%%


%Each node has a single channel on which it listens. In order to maintain a link when a node changes its channel, all neighbours must be informed of the change. We use two-hop colouring protocol to avoid collision. 
%Our protocol is inspired by the routing protocol for low power and lossy networks (RPL). Packets will be sent to the destination the same way as a single channel RPL but with less loss. 
%All nodes are battery operated except for the low power border route (LPBR). This enables a centralised channel switching process at the LPBR. 
%%%%The channel switching process take place after the topology is formed to further improve the transmission rate on the best paths.
%In its initial phase, the protocol uses RPL's standard topology formation to create an initial working topology and then seeks to improve this topology by switching channels.
%We implement and evaluate our solution using the Contiki framework. Our experimental results demonstrate an increased resilience to interference, and significant higher throughput making better use of the total available spectrum and link stability. 
