\section{Introduction}
\label{sec:introduction}
Wireless Sensor Network (WSN) is an ad-hoc network that consists of sensor nodes that typically use low power radios such as IEEE 802.15.4, a relatively short range transmission standard radio technology in the 2.4 GHz band. The standard allows transmission to occur on several different channels within this band \cite{ieee802.15.4}. Unfortunately, the channels used by this technology often suffer interference \cite{Boano:2010:MSM:2127940.2127963, ieeeCompare}, for example, from Wi-Fi \cite{ieee_2012, wu} and Bluetooth. Sensor networks have to contend with an increasing number of devices that cause this wireless interference. Organising the network topology around this interference becomes an enabler for increasing transmission efficiency at a smaller energy cost. WSNs need to be able to operate reliably in the presence of such interference. It is important to minimise energy costs in these networks since deployments can be for weeks, months or longer.

Multichannel communication in wireless networks can alleviate the effects of interference which, as a result, can improve the network efficiency and stability, link reliability and minimise latency \cite{watteyne}. It also enables communication between physically proximate nodes to occur simultaneously without the risk of collision when the communicating nodes use different channels. However, not all channels are free from interference; thus, there is a need to hop to another channel when the quality of the channel deteriorates. Two commonly used types of channel hopping \cite{watteyne} are blind channel hopping and whitelisting. In blind channel hopping, nodes choose from all available channels. 
Whitelisting, on the other hand, gives a set list of channels that avoids those that are known to commonly suffer interference.
Many studies make use of channel whitelisting such as in Chrysso \cite{chrysso} and MiCMAC \cite{micmac}. However, they used different channels with channel 26 in common in their respective experiments. 

Note that potentially Chrysso and MiCMAC could use all available channels.
However, they do not have a mechanism to check the channel condition before using it for packet transmission. MiCMAC sees its performance degraded when using more than 4 channels, thus the decision on specifying 4 channels to be included in their experiment. 
MiCMAC uses a different channel chosen at random each time it wakes up.
It might requires several wake up periods which is time consuming, before a clear channel is found from the 16 channels, to delivery the packet.  
Chrysso on the other hand, switches the affected nodes to a new set of channels upon detecting interference which entails frequent channel switching if all channels are to be considered.

It is clear that it is impossible to find a single channel guaranteed free from interference and there is no consensus on the best channel to use. Our work takes into account all available channels to utilise the spectrum and checks the condition of the channels before hopping to avoid those channels with interference. Several previous studies have developed a multichannel MAC layer but, despite the potential benefits none are yet widely implemented in real world deployments. 
A possible explanation for this is that the existing solutions are too complex to implement and use, and require transmissions scheduling or synchronisation. 

This paper focuses on a Multichannel Cross-Layer Routing Protocol (MCRP) which consists of two main parts; a centralised intelligence at LPBR, and decentralised nodes. LPBR implements a two-hop colouring protocol to avoid interference between physically proximate nodes trying to communicate on the same channel. The information on channel interference and network topology from the lower layer is made available to the application layer. This allows the centralised controller (LPBR) to have an overall view of the system to make decisions at the network and MAC layers about which channels nodes should listen on. The system is fail safe in the sense that the WSN functions if the central system which assigns channels fails temporarily or permanently.

We implement MCRP in Contiki \cite{contiki}, an open source operating system for the Internet of Things and evaluate the protocol in Contiki network simulator, Cooja \cite{cooja}. 
%and in 26-nodes testbed FlockLab \cite{flocklab}. 
We demonstrate that MCRP avoids channels with interference which greatly reduces the effects of interference on the network.

The rest of the paper is organised as follows: Section \ref{sec:relatedwork} presents related work to multichannel protocols. Section \ref{sec:multichannel} describes the key idea of our proposed protocol and the high-level design and the implementation of the protocol in Contiki. We describe and evaluate the experimental results in Section \ref{sec:evaluation}. Finally, we conclude in Section \ref{sec:conclusion}.