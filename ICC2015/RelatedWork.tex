\section{Related Work}
\label{sec:relatedwork}
Radio duty cycling mechanisms can be classified into two categories; synchronous and asynchronous systems. A synchronous system is a system that requires a tight time synchronisation between the nodes. It uses time-scheduled communication where the network clock needs to be periodically synchronised in order for the nodes not to drift in time. Asynchronous systems on the other hand, do not require synchronisation but instead is a sender or receiver initiated communication. In asynchronous systems the nodes are able to self-configure without time synchronisation and this can have advantages. There are many studies done in multichannel for both categories. Multichannel synchronous protocols include MC-LMAC \cite{mc-lmac}, Y-MAC \cite{y-mac}, and TSCH \cite{tsch}. Multichannel asynchronous protocols such as EM-MAC \cite{emmac}, MuChMAC \cite{muchmac}, Chrysso \cite{chrysso}, MiCMAC \cite{micmac} and our protocol are independent of time slot and synchronisation. 

MiCMAC \cite{micmac} is a ContikiMAC \cite{contikimac} channel hopping variant. ContikiMAC was proved to be efficient in \cite{micmac,orpl} for a single channel. In MiCMAC, on every wakeup cycle, the channel is periodically switched according to a pseudo-random sequence. Chrysso \cite{chrysso} is a multichannel protocol for data collection applications. The nodes are organised into parent-children groups where each parent-children uses two channels for transmitting and receiving packets. MCRP also uses two separate channels as in Chrysso. In Chrysso, both parent and children nodes can hop to another channel when interference is detected based on the channel switching policies. Chrysso functionality comprises a set of channel switching policies that interface to both the MAC layer and the network layer. 

MiCMAC and Chrysso are fully distributed and allow the nodes to self configure and change to another channel when interference happens. The channels that the protocols can use are fixed to a subset of whitelisted channels. MiCMAC and Chrysso could use all channels, however, they do not have a mechanism for channel quality checking before a channel is chosen. It would be time consuming before it could find the interference free channel. This contrasts with our protocol where we are able to produce real time channel selection decisions by considering all available channels to be used in transmissions without blacklisting any of them. 

In order to maximise the use of multichannel in improving packet delivery, routing topology plays a big role in providing an optimised routing tree to the network that is scalable and energy efficient. There have been many studies on routing protocol such as LEACH \cite{leach}, PEGASIS \cite{pegasis}, CTP \cite{ctp}, and RPL \cite{winter2012rpl}. Recent multichannel protocols such as MiCMAC is compatible with RPL as the routing protocol. Chrysso uses Contiki collect which is a CTP-like data collection protocol in Contiki. Chrysso is restricted to only data collection networks. We choose to use RPL as it is the standard for IPv6 routing in low power and lossy networks. RPL \cite{winter2012rpl, routingmetrics, mrhof} is a gradient based routing protocol forming any-to-any routing for low power IPv6 networks. Our protocol makes use of RPL topology formation and improves on the channels of the nodes in the topology.