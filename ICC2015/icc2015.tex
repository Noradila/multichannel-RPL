%% icc2015.tex
\documentclass[conference]{IEEEtran}

\usepackage[utf8]{inputenc}
\usepackage[pdftex]{graphicx}
\usepackage{algorithm}
\usepackage{algorithmic}
\usepackage{subfigure}
\usepackage{url}

\ifCLASSINFOpdf
  % \usepackage[pdftex]{graphicx}
  % declare the path(s) where your graphic files are
  % \graphicspath{{../pdf/}{../jpeg/}}
  % and their extensions so you won't have to specify these with
  % every instance of \includegraphics
  % \DeclareGraphicsExtensions{.pdf,.jpeg,.png}
\else
  % or other class option (dvipsone, dvipdf, if not using dvips). graphicx
  % will default to the driver specified in the system graphics.cfg if no
  % driver is specified.
  % \usepackage[dvips]{graphicx}
  % declare the path(s) where your graphic files are
  % \graphicspath{{../eps/}}
  % and their extensions so you won't have to specify these with
  % every instance of \includegraphics
  % \DeclareGraphicsExtensions{.eps}
\fi
% graphicx was written by David Carlisle and Sebastian Rahtz. It is
% required if you want graphics, photos, etc. graphicx.sty is already
% installed on most LaTeX systems. The latest version and documentation
% can be obtained at: 
% http://www.ctan.org/tex-archive/macros/latex/required/graphics/
% Another good source of documentation is "Using Imported Graphics in
% LaTeX2e" by Keith Reckdahl which can be found at:
% http://www.ctan.org/tex-archive/info/epslatex/
%
% latex, and pdflatex in dvi mode, support graphics in encapsulated
% postscript (.eps) format. pdflatex in pdf mode supports graphics
% in .pdf, .jpeg, .png and .mps (metapost) formats. Users should ensure
% that all non-photo figures use a vector format (.eps, .pdf, .mps) and
% not a bitmapped formats (.jpeg, .png). IEEE frowns on bitmapped formats
% which can result in "jaggedy"/blurry rendering of lines and letters as
% well as large increases in file sizes.
%
% You can find documentation about the pdfTeX application at:
% http://www.tug.org/applications/pdftex


% correct bad hyphenation here
\hyphenation{op-tical net-works semi-conduc-tor}


\begin{document}
%
% paper title
% Titles are generally capitalized except for words such as a, an, and, as,
% at, but, by, for, in, nor, of, on, or, the, to and up, which are usually
% not capitalized unless they are the first or last word of the title.
% Linebreaks \\ can be used within to get better formatting as desired.
% Do not put math or special symbols in the title.
\title{Multichannel Cross-Layer Routing for Sensor Networks}


% author names and affiliations
% use a multiple column layout for up to three different
% affiliations
\author{\IEEEauthorblockN{Noradila Nordin}
\IEEEauthorblockA{
University College London\\
%Electronic and Electrical Engineering Department\\ University College London\\ 
%London, WC1E 7JE\\
Email: noradila.nordin.12@ucl.ac.uk}
\and
\IEEEauthorblockN{Richard G Clegg}
\IEEEauthorblockA{Imperial College London\\
Email: richard@richardclegg.org}
\and
\IEEEauthorblockN{Miguel Rio}
\IEEEauthorblockA{University College London\\
Email: miguel.rio@ucl.ac.uk}}

% make the title area
\maketitle

% As a general rule, do not put math, special symbols or citations
% in the abstract
\begin{abstract}
%The abstract goes here.
This paper proposes a new multi-channel tree building protocol for ad-hoc sensor networks. Our protocol alleviates the effect of interference which results in improved network efficiency and stability, link reliability and minimised latency. 
        Our proposal takes into account all available channels to utilise the spectrum. Our protocol aims to use the spectrum efficiently by transmitting on several channels. The protocol detects which channels suffer interference and changes away from those channels.
%It checks the condition of all the channels before deciding on a channel to switch into. The successful transmission rate of the channels are stored externally from the sensors which can be accessed when require. 
This information is used to limit the channels to be considered when channel switching is invoked. The channel that is selected is checked for any changes in its condition that might had taken place after it was checked previously before committing to the channel. The results and decisions are informed to the other nodes to update their neighbour table. We use two-hop colouring protocol to avoid collision. 
	Our protocol is inspired by the routing protocol for low power and lossy networks (RPL). Packets will be sent to the destination the same way as a single channel RPL but with less loss. 
	All nodes are battery operated except for the low power border route (LPBR). This enables a centralised channel switching process at the LPBR. The channel switching process take place after the topology is formed to further improve the transmission rate on the best paths.
	We implement and evaluate our solution using the Contiki framework. Our experimental results demonstrate an increased resilience to interference, and significant higher throughput making better use of the total available spectrum and link stability. 

\end{abstract}
% no keywords




% For peer review papers, you can put extra information on the cover
% page as needed:
% \ifCLASSOPTIONpeerreview
% \begin{center} \bfseries EDICS Category: 3-BBND \end{center}
% \fi
%
% For peerreview papers, this IEEEtran command inserts a page break and
% creates the second title. It will be ignored for other modes.
\IEEEpeerreviewmaketitle

\section{Introduction}
\label{sec:introduction}
Wireless Sensor Networks (WSN) are ad-hoc networks that consist of sensor nodes that typically use low power radios such as IEEE 802.15.4, a relatively short range transmission standard radio technology in the 2.4 GHz band. The standard allows transmission to occur on several different channels within this band \cite{ieee802.15.4}. Unfortunately, the channels used by this technology often suffer interference \cite{Boano:2010:MSM:2127940.2127963, ieeeCompare}, for example, from Wi-Fi \cite{ieee_2012, wu} and Bluetooth. Sensor networks have to contend with an increasing number of devices that cause this wireless interference. Organising the network topology around this interference becomes an enabler for increasing transmission efficiency at a smaller energy cost. WSNs need to be able to operate reliably in the presence of such interference. It is important to minimise energy costs in these networks since deployments can be for weeks, months or longer.

Multichannel communication in wireless networks can alleviate the effects of interference which, as a result, can improve the network efficiency and stability, link reliability and minimise latency \cite{watteyne}. It also enables communication between physically proximate nodes to occur simultaneously without the risk of collision when the communicating nodes use different channels. However, not all channels are free from interference; thus, there is a gain to hop to another channel when the quality of the channel deteriorates. Two commonly used types of channel hopping \cite{watteyne} are blind channel hopping and whitelisting. In blind channel hopping, nodes choose from all available channels. 
Whitelisting, on the other hand, gives a set list of channels that avoids those that are known to commonly suffer interference.
Many studies make use of channel whitelisting such as in Chrysso \cite{chrysso} and MiCMAC \cite{micmac}.

Note that potentially Chrysso and MiCMAC could use all available channels.
However, they do not have a mechanism to check the channel condition before using it for packet transmission. MiCMAC sees its performance degraded when using more than 4 channels, thus the decision on specifying 4 channels to be included in their experiment. 
MiCMAC uses a different channel chosen at random each time it wakes up.
It might require several wake up periods which is time consuming, before a clear channel is found from the 16 channels, to deliver the packet.  
Chrysso on the other hand, switches the affected nodes to a new set of channels upon detecting interference which entails frequent channel switching if all channels are to be considered.

It is clear that it is impossible to find a single channel guaranteed free from interference and there is no consensus on the best channel to use. Our work takes into account all available channels to utilise the spectrum and checks the condition of the channels before hopping to avoid those channels with interference. Several previous studies have developed a multichannel MAC layer but, despite the potential benefits none are yet widely implemented in real world deployments.

This paper presents a Multichannel Cross-Layer Routing Protocol (MCRP) which consists of two main parts; a centralised intelligence at LPBR, and decentralised nodes. LPBR implements a two-hop colouring algorithm to avoid interference between physically proximate nodes trying to communicate on the same channel. The information on channel interference and network topology from the lower layer is made available to the application layer. This allows the centralised controller (LPBR) to have an overall view of the system to make decisions at the network and MAC layers about which channels nodes should listen on. The system is fail safe in the sense that the WSN functions if the central system which assigns channels fails temporarily or permanently.

We implement MCRP in Contiki \cite{contiki}, an open source operating system for WSNs and evaluate the protocol in Contiki network simulator, Cooja \cite{cooja}. 
We demonstrate that MCRP avoids channels with interference which greatly reduces the effects of interference on the network.

The rest of the paper is organised as follows: Section \ref{sec:relatedwork} presents related work to multichannel protocols. Section \ref{sec:multichannel} describes the key idea of our proposed protocol and the high-level design, and the implementation of the protocol in Contiki. We describe and evaluate the experimental results in Section \ref{sec:evaluation}. Finally, we conclude in Section \ref{sec:conclusion}.


\section{Related Work}
%- multichannel enables communication between more than one node to occur simultaneously and in a collision free manner

%\subsection{Existing Multichannel Protocol}
Radio duty cycling mechanism can be classified into two categories; synchronous and asynchronous system. A synchronous system is a system that requires a tight synchronisation between the nodes. It uses a tight time-scheduled communication where the network clock needs to be periodically synchronise in order for the nodes not to drift in time. Asynchronous multichannel on the other hand, does not require synchronisation but instead, is a sender or receiver initiated communications. It requires simple set up and the nodes are able to self-configure without tight synchronisation which is more appealing. There are many studies done in multichannel for both categories. Multichannel synchronous protocols for such as MC-LMAC \cite{mc-lmac} which uses time slot to transmit on a particular channel and Y-MAC \cite{y-mac}, EM-MAC \cite{emmac} and TSCH that depend on the neighbouring nodes to synchronise with each other. Multichannel asynchronous protocols such as MuChMAC \cite{muchmac}, Chrysso \cite{chrysso}, MiCMAC \cite{micmac} and our protocol are independent of time slot and synchronisation. 

%//!!why asynchronous vs synchronous important?
%This is important as in synchronous multichannel, the nodes need to be globally synchronise which is hard to achieve with frequent channel changes as the nodes need to maintain a tight synchronisation and need to synchronise the network clock periodically not to drift in time. Need to have a time-scheduled communication. Asynchronous multichannel on the other hand, can be either sender or receiver initiated communications. This shifts the global synchronise to local and does not affect the whole network.-simple setup, self-configurable as it does not need tight synchronisation.

ContikiMAC \cite{contikimac} is the default radio duty cycling protocol in Contiki that is responsible for the node wake-ups period. ContikiMAC is a power-saving radio duty cycling protocol. It was proved to be efficient in a single channel \cite{micmac}\cite{orpl}. ContikiMAC uses periodical wake-ups to listen to the neighbours transmission packet. It has a phase-lock mechanism to learn the neighbours wake-up phase to enable efficient transmissions and a fast sleep optimisation in case of spurious radio interference is detected. The sender uses the knowledge of the wake-up phase of the receiver to optimise its transmission. When a packet is successfully received, the receiver sends a link layer acknowledgement. The sender repeatedly sends its packet until it receives a link layer acknowledgement from the receiver. ContikiMAC relies on retransmissions for reliable transmissions. A Carrier Sense Multiple Access, CSMA is a MAC protocol that performs retransmissions when the underlying MAC layer has problems with collisions. When the sender does not receive the link layer acknowledgement, CSMA will retransmit the packets three times before dropping it from the buffer queue.
%CSMA is the MAC layer that takes care of retransmission of lost packets. 
%A packet is consider as loss when the sender does not receive the link layer acknowledgement. 


%//loss and retransmission - how it deals with it

%//be clear about when protocol suffers a loss

MiCMAC \cite{micmac} is an asynchronous protocol, ContikiMAC \cite{contikimac} channel hopping variant. On every wakeup cycle, the channel is periodically switched according to a pseudo-random sequence. MiCMAC introduces channel lock for the channel reception at the sender. There is a dedicated broadcast channel for a duration at every wake up period. %MiCMAC can be used with RPL without any changes to RPL. %However, RPL might not formed properly because the nodes need to be on the broadcast channel at different time for different node according to the trickle timer to receive RPL control messages which are sent through unicast and broadcast.

Chrysso \cite{chrysso} is a multichannel protocol for data collection applications. The nodes are organised into parent-children groups where each parent-children uses two channels for transmitting and receiving packets. Our work also uses two channels as in Chrysso. Both parent and children nodes can hop to another channel when interference is detected based on the channel switching policies. If a node loses connectivity, Chrysso calls the scan mode to enable neighbour discovery over multiple channel. Chrysso’s functionality comprises a set of channel switching policies that interface to both the MAC layer and the network layer.

However, MiCMAC and Chrysso are fully distributed which allows the nodes to self configure and change to another channel when interference happen. Our protocol is centralised where most of the processes in channel assignment decisions are done by the LPBR. LPBR, which is the central point, is fully powered and does not have limited memory. We are able to produce real time channel selection decisions where we can consider all available channels to be used in transmissions without blacklisting any of them.  

//distributed vs centralised
Most of the (available?) multichannel protocols are fully distributed (self configuration?) and does not has a centralised system. This is the way it should be as sensors have limited battery life and memory but centralised would enable real time decisions more efficient - moving decisions making to a central point and reduce unnecessary processing on the nodes.

MiCMAC can be used with RPL without any changes to RPL.
 
%//(Collect)//?Chrysso concentrates on data collection while our work tries to improve RPL single channel into multichannel without dealing with the MAC layer. RPL is typically used in ContikiMAC which is a single channel.

In order to maximise the good use of (???) multichannel, routing topology play a big role - for scalability, to save energy (efficient) while sending on an optimised routes, maximise the chance of packets being received (minimise loss).

RPL do topology maintenance*.

There are many studies that were done on routing protocol such as tree based LEACH \cite{leach}, PEGASIS \cite{pegasis}, CTP \cite{ctp} and RPL which is designed largely based on CTP.

Dynamic/static routing topology?


However, only recent multichannel protocols such as MiCMAC uses RPL as the routing protocol. Chrysso uses Contiki collect which is a CTP-like data collection protocol in Contiki. We choose to use RPL as it is the standard for IPv6 routing in low power and lossy networks. RPL forms the network topology dynamically.

%routing in RPL
Routing protocol for low power and lossy networks (RPL) is a gradient based routing protocol forming any-to-any routing for low power IPv6 networks. RPL topology is a Destination-Oriented Directed Acyclic Graph (DODAG), rooted at LPBR with no cycles. The root has the overall view of the network. The other nodes however, only has knowledge of its neighbours and default router. RPL is a rooted topology which any-to-any traffic is directed towards the root unless the common ancestor is found which the traffic is then routed downwards towards the destination. This strategy is used in order to scale large networks by reducing the routing overhead at the cost of increased hop count through common ancestor. 

In RPL terminology, the node distance to the root and other nodes is defined as the node's rank. RPL finds the path with the minimum number of transmissions that a node expect to successfully deliver a packet to the destination and switches only if it is less than the current rank to prevent frequent changes~\cite{mrhof}. 

%**NOT YET ADDED - Multichannel protocol such as MC-LMAC, Y-MAC, MuChMAC, EM-MAC. What these works do?

%TSCH (Time Synchronized Channel Hopping) employs TDMA and channel hopping such that it schedules communications in 2 dimensions: time and frequency.

%Timeslotted Channel Hopping (TSCH) \cite{tsch} focuses on the MAC layer which can be used with RPL. It uses time synchronisation and channel hopping for ultra low power operation and high reliability. TSCH schedules the communication in time and frequency slot. However, it requires tight synchronisation to the neighbour nodes. In order to use TSCH with RPL, there are several problems to be addressed in term of the topology and network maintenance as RPL determines the multihop route and has it's own control messages.

%Chrysso is a multi-channel solution that is specifically designed for mitigating external interference in data collection WSNs. The core of Chrysso’s functionality comprises a set of channel switching policies that interface to both the MAC layer and the network layer (Collect).


Our proposed protocol takes into account RPL topology formation scheme and the control messages exchange between nodes that take place frequently to maintain the quality of the tree. The RPL control messages are sent through unicast in order to reduce unnecessary transmitting in broadcast. Our work makes use of RPL topology formation and improves on the channel within the topology formed. The nodes do not need to sync with one another as they would know the listening channel of the other nodes.

//why ours is better?? compare with MiCMAC, Chrysso, MuChMac. "We are different because because we are cross layer with a centralised co-ordinator. RPL is typically used with ContikiMAC, a single channel protocol"

We don't blacklist any channels; checks before using the channel. Centralised channel change decision; at LPBR. Nodes decide if the channel recheck LPBR decision of the channel and change. LPBR has the intelligence in choosing the channel for the node as it has the full overview of the condition of all nodes based on a periodic information from the nodes. LPBR will have the information of the good and bad channel for each node. 
Cross layer; channel changes on network layer while others use layer 2 (?), the MAC layer. This enables us to do cleverly? and deciding the channel cross layer - can do complicated decisions than if with MAC layer channel.


%- channel? repetition of time slot over 2-hop range? TDMA-W?
%- S-MAC? contend for the channel during listen cycles? meaning?
%- pair wise synchronisation?
%- energy savings - obtained by allowing radio for data communication to enter the low power sleep mode
%- ContikiMAC + CSMA - it does the backoff (solve deafness problem?) check how CSMA deals with deafness. When deafness happen in our case?
%- overhearing avoided as nodes are assigned different channels
%- transmit at a constant power level?
%- nodes are allocated channels that overlap at 3 hops or more through the execution of a suitable channel assignment algorithm
%- channel negotiation is only when selecting channel to listen on - based on LPBR channel change message
%- collision - if nodes try to send to the same parent/node. Collision won't happen even during the phase for control channel exchange (receive channel change from LPBR) since it won't happen all at once

%\subsection{RPL Topology}

%routing in RPL
%Routing protocol for low power and lossy networks (RPL) is a gradient based routing protocol forming any-to-any routing for low power IPv6 networks. RPL topology is a Destination-Oriented Directed Acyclic Graph (DODAG), rooted at a single destination, called the Low Power and lossy network Border Router (LPBR) with no cycles. The root has the overall view of the network. The other nodes however, only has knowledge of its neighbours and default router. RPL is a rooted topology which any-to-any traffic is directed towards the root unless the common ancestor is found which the traffic is then routed downwards towards the destination. This strategy is used in order to scale large networks by reducing the routing overhead at the cost of increased hop count through common ancestor. However, RPL has drawbacks. It takes a while before a broken link is detected and global repair to take place.

%objective function
%routing metrics
%RPL separates the routing optimisation objectives with the packet processing and forwarding policies. This allows the networks to have different optimisation objectives which include the rank computation, node selection, parents selection and route optimisation. However, RPL Objective Function (OF) is not an algorithm. It is the process to optimise the routes. In RPL terminology, the node distance to the root and other nodes is defined as the node's rank. The rank increases downwards and decreases towards the root. There are two Objective Functions specified as the standard, which are the Objective Function Zero (OF0) and Minimum Rank with Hysteresis Objective Function (MRHOF). OF0 is based on the shortest amount of hops to the root~\cite{of0}. MRHOF uses hysteresis with the Expected Transmission Count (ETX) metric. ETX is the number of transmissions that a node expect to successfully deliver a packet to the destination. It finds the path with the minimum rank and switches only if the rank is less than the current rank to prevent frequent changes~\cite{mrhof}. Hop count and ETX are node and link metrics which is a direct conversion of metric to rank values. By default, Contiki uses MRHOF.

%rpl messages
%RPL defines three main types of ICMPv6 based control messages used in topology formation and maintenance which are the DODAG Information Object (DIO), Destination Advertisement Object (DAO) and DODAG Information Solicitation (DIS). DIO contains the information needed by the node such as the configuration parameters, parent selection and rank. DAO is used to enable packets to propagate towards the root. DIS is used by a node to require  DIO messages from its neighbour in order to join the DODAG. 

%topology formation
%trickle timer
%The topology starts that the root node that sends DIO messages to its neighbours. The nodes that receive the message choose it's parent based on the OF, updates the DIO message before sending to its neighbours. The same processes are repeated until all nodes have joined the tree. DIO message is also used for topology maintenance. However, calling the messages frequently would make the network congested. Thus, Trickle timer is used. Trickle algorithm controls the transmission rate of DIO messages. When the timer expires, Trickle doubles the timer interval until it reaches the maximum value. However, when an event occurs, the timer is reset to the minimum value~\cite{trickle}. By default, Contiki uses $2^{12}$ milliseconds as the minimum interval between DIO messages and $2^{20}$ milliseconds as the maximum value.

\section{Multichannel Cross-Layer Routing Protocol}
\label{sec:multichannel}

Please consider the following notes on the design:
- What if the current channel that you are using to communicate channel-switch decision is  too bad that you could not get the message through?
- Channel conditions change rapidly with time. What about the communication time from the central coordinator to the node?

* The authors state that channel quality checking is performed by sending 8
  packets to each neighbor. Apart from the evaluation of this overhead, the
  authors fail to state the inter-packet interval, which (if too short) may
  have an effect on the estimate. Also, the threshold of 7 appears to be quite
  high. What happens if no neighbors meets the threshold (see above examples)?
  
  

\section{Evaluation}
%ours, RPL + ContikiMAC, MiCMAC
The results of our multichannel RPL protocol is compared against single-channel tree protocol and MiCMAC.

\subsection{Experimental Setup}
%//methodology? key metrics?
We evaluate the protocol in Cooja simulated environment with emulation of TMote sky nodes that feature the CC2420 transceiver, a 802.15.4 radio. The nodes run on IPv6, using UDP with standard RPL and 6LoWPAN protocols. The network consists of 16 nodes are used to run the simulation where we have 1 border router node, 1 interference node, and 14 duty cycled nodes that act as UDP clients to send packets to LPBR. We simulated a controlled interference node that generates semi-periodic bursty interference to resemble a simplified WiFi or Bluetooth transmitter on channel 26; which is the channel LPBR is listening on.

//!!!why 1 interference only?
The simulator, Cooja has a memory leak problem which unable us to use more interference on different channels as we would run out of memory to run the simulation for 60 minutes. 

RPL border router is used in order to move most processing decisions on a PC as it has more RAM and better processing capabilities than a sensor. TelosB has limited RAM and ROM of (ram value?), (rom value). By using a border router, this allows channel changing to be decided in real time. 

%RPL border router is used to interface a regular IPv6 with a regular RPL.

%(explain border router etc????) that always stays on and it is the sink of the tree, also the root node of RPL. There are 14 duty cycled nodes that acts as UDP clients that send packets to LPBR. We simulated a controlled interference node that generates semi-periodic bursty interference on channel 26; which is the channel LPBR is listening on. 

%several channels, mainly on the parents channel as the packets travel upward to measure the benefits of multi-channel operation of changing channel when failure happens within a given period. 

%- throughput?

We evaluate multichannel RPL variant using two performance metrics: end to end packet delivery and latency. In end to end packet delivery, the transmission success rate is calculated from the sender to the receiver over multiple hops. The latency, time difference from sending to receiving is also calculated based on Cooja log time. %We compare our multichannel protocol with Contiki's default ContikiMAC and RPL, and MiCMAC.

We run the simulation for a duration of 60 minutes to send 700 packets; 50 packets for each node. RPL runs the initial network setup for a few minutes before it is stable. We set the RPL setup time to be 5 minutes before our multichannel protocol runs for 10 minutes. After 15 minutes, the client nodes will send a normal packet periodically every 30-60 seconds to LPBR. This is done in order to avoid collision of the nodes sending at the same time. The simulation is repeated 3 times.

%We evaluate multichannel RPL variant using three performance metrics: end to end packet delivery, latency and duty cycle. In end to end packet delivery, the transmission success rate is calculated from the sender to the receiver over multiple hops. The latency, time difference from sending to receiving is also calculated based on Cooja log time. Contiki's energy profiler is used to measure the duty cycle where the radio usage time in the total run is calculated.



\subsection{Effect of Multi-channel}
//with existings - better? worse? what about RAM, ROM used?

\subsection{Resilience to External Interference}

\section{Conclusion}
\label{sec:conclusion}
%//conclusion and future work

%//"Work is continuing to develop this protocol.  The next stages are 1) to improve the interference model in testing to a multiple channel interference model; 2) To move the implemenation to real hardware and 3) To allow continual updates on packet loss for each node so that channels can be changed dynamically when interference occurs."

We presented MCRP, a centralised cross-layer protocol that mitigates the effect of interference by avoiding affected channels and allows better spectrum usage by trying to move nearby nodes to listen on different channel. The interference avoidance is through probing when moving to a new channel. The results from the simulation showed that our protocol avoids channels with interference and hence avoids packet loss.  We also showed that this system works best when trying to avoid extreme interference

Future work is ongoing to develop the protocol. The next stages that we plan to pursue is to improve the interference model that we used in testing to cover multiple interference channels replicating the real world environment. The protocol will be tested against competing multi-channel protocols such as MiCMAC. We also plan to test our implementation on real hardware.  Finally we will allow nodes to update the LPBR on packet loss experienced in order that changes to interference patterns in the network can be reacted to.

% The nature of our protocol that decides/run the decision making to the application layer gives us more opportunity to add the decision complexity - intelligence? and do real time channel checking. Our protocol maintains high reliability during heavily interfered periods where  ContikiMAC showed a low throughput of //////result?.



\section*{Acknowledgments}
Noradila Nordin is a King's Scholar sponsored by the Government of Malaysia.

\label{references}
\nocite{*}
\bibliography{icc2015}
\bibliographystyle{plain}

\end{document}