\section{Conclusion}
\label{sec:conclusion}

%We presented MCRP, a cross-layer protocol that consists of a centralised control node (LPBR) that uses a two-hop colouring protocol for deciding channels that nodes should change into; and decentralised nodes which allows nodes to determine whether or not they should change to a new listening channel through probing.

We presented MCRP, a centralised cross-layer protocol that mitigates the effect of interference by avoiding affected channels and allows better spectrum usage by trying to move nearby nodes to listen on different channel using two-hop colouring protocol. Our protocol provides feedback when a channel is subject to interference using a probing phase.
%The interference avoidance is through probing when moving to a new channel.
The results from the simulation showed that our protocol avoids channels with interference hence has a high packet receiving rate.

%avoids packet loss.

%Future work is ongoing to develop the protocol. The next stages that we plan to pursue is to improve the interference model that we used in testing to cover multiple interference channels replicating the real world environment. The protocol will be tested against competing multi-channel protocols such as MiCMAC. We also plan to test our implementation on real hardware.  Finally we will allow nodes to update the LPBR on packet loss experienced in order that changes to interference patterns in the network can be reacted to.

